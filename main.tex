\documentclass[a4paper,12pt]{article}
\usepackage[utf8]{inputenc}
\usepackage[T2A]{fontenc}
\usepackage[russian]{babel}
\usepackage{amsmath,amsfonts}
\usepackage{graphicx}
\usepackage{booktabs}
\usepackage{geometry}
\geometry{margin=2.5cm}
\usepackage{float}
\usepackage{pgfplots}
\pgfplotsset{compat=1.17}

\title{Численное решение 1D-уравнения Лапласа}
\author{Харитонов Фёдор, 203}
\date{Апрель 2025}

\begin{document}

\maketitle

\section*{Постановка задачи}
Рассматривается краевая задача Дирихле для уравнения Лапласа:
\[ -u''(x) = f(x), \quad x \in (0,1), \quad u(0) = a, \quad u(1) = b, \]
где точное решение выбрано как \( u(x) = \sin(x) \). Тогда
\[ f(x) = \sin(x), \quad a = 0, \quad b = \sin(1). \]

Цель — численно найти приближённое решение задачи с использованием метода конечных разностей и оценить сходимость по \( C \)- и дискретной \( L_2 \)-нормам.

\section*{Метод численного решения}
Интервал \((0,1)\) разбивается на равномерную сетку:
\[ x_i = ih, \quad h = \frac{1}{N}, \quad i = 0, \dots, N. \]
Искомые значения \( y_i \approx u(x_i) \). Аппроксимация:
\[ -\frac{y_{i-1} - 2y_i + y_{i+1}}{h^2} = f(x_i), \quad i = 1, \dots, N-1. \]
Получается линейная система с трёхдиагональной матрицей, решаемая методом прогонки.

\section*{Результаты численных экспериментов}

\begin{table}[H]
\centering
\begin{tabular}{@{}ccccc@{}}
\toprule
$N$ & $h$ & $C$-норма ошибки & $L_2$-норма ошибки & Время (сек.) \\
\midrule
\input{errors_table.tex}
\bottomrule
\end{tabular}
\caption{Сходимость численного решения при разных $N$}
\end{table}

\section*{Графики}

\begin{figure}[H]
\centering
\begin{tikzpicture}
\begin{loglogaxis}[
    xlabel={$h$}, ylabel={Ошибка},
    legend pos=south east,
    grid=major,
    width=0.75\textwidth, height=0.5\textwidth
]
\addplot table [x=h, y=C_norm, col sep=comma] {errors.csv};
\addplot table [x=h, y=L2_norm, col sep=comma] {errors.csv};
\legend{$C$-норма, $L_2$-норма}
\end{loglogaxis}
\end{tikzpicture}
\caption{Зависимость ошибки от шага сетки}
\end{figure}

\begin{figure}[H]
\centering
\begin{tikzpicture}
\begin{semilogxaxis}[
    xlabel={$N$}, ylabel={Время (сек.)},
    legend pos=north west,
    grid=major,
    width=0.75\textwidth, height=0.5\textwidth
]
\addplot table [x=N, y=time, col sep=comma] {errors.csv};
\legend{Время вычислений}
\end{semilogxaxis}
\end{tikzpicture}
\caption{Зависимость времени вычислений от числа узлов $N$}
\end{figure}

\section*{Выводы}
Наблюдаем квадратичную сходимость численного метода, что соответствует ожидаемой точности \( O(h^2) \) аппроксимации второй производной. Время вычислений растёт линейно с увеличением числа узлов $N$.

\end{document}